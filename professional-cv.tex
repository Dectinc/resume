% !TEX program = XeLaTeX
%!TEX TS-program = xelatex

\documentclass[a4paper,10pt]{professional-cv-cn}
\usepackage{fontawesome}

%A Few Useful Packages
\usepackage{marvosym}
% \usepackage{fontspec} 					%for loading fonts
\usepackage{xunicode,xltxtra,url,parskip} 	%other packages for formatting
\RequirePackage{color,graphicx}
\usepackage[usenames,dvipsnames]{xcolor}
\usepackage[big]{layaureo} 				%better formatting of the A4 page
% an alternative to Layaureo can be ** \usepackage{fullpage} **
\usepackage{supertabular} 				%for Grades
\usepackage{titlesec}					%custom \section

%Setup hyperref package, and colours for links
\usepackage{hyperref}
\definecolor{linkcolour}{rgb}{0,0,0}
\hypersetup{colorlinks,breaklinks,urlcolor=linkcolour, linkcolor=linkcolour}

%CV Sections inspired by:
%http://stefano.italians.nl/archives/26
\titleformat{\section}{\Large\scshape\raggedright}{}{0em}{}[\titlerule]
\titlespacing{\section}{0pt}{2pt}{2pt}
%Tweak a bit the top margin
%\addtolength{\voffset}{-1.3cm}

%Italian hyphenation for the word: ''corporations''
\hyphenation{im-pre-se}

%-------------WATERMARK TEST [**not part of a CV**]---------------
\usepackage[absolute]{textpos}
\usepackage{multirow}

% \setlength{\TPHorizModule}{30mm}
% \setlength{\TPVertModule}{\TPHorizModule}
% \textblockorigin{2mm}{0.65\paperheight}
% \setlength{\parindent}{0pt}

%--------------------BEGIN DOCUMENT----------------------
\begin{document}

%WATERMARK TEST [**not part of a CV**]---------------
%\font\wm=''Baskerville:color=787878'' at 8pt
%\font\wmweb=''Baskerville:color=FF1493'' at 8pt
%{\wm
%	\begin{textblock}{1}(0,0)
%		\rotatebox{-90}{\parbox{500mm}{
%			Typeset by Alessandro Plasmati with \XeTeX\  \today\ for
%			{\wmweb \href{http://www.aleplasmati.comuv.com}{aleplasmati.comuv.com}}
%		}
%	}
%	\end{textblock}
%}

\pagestyle{empty} % non-numbered pages

\font\fb=''[cmr10]'' %for use with \LaTeX command

%--------------------TITLE-------------
\begin{tabular}{lr}
\multirow{4}{24em}{\Huge ~~~~~~~~~陈仕江}
    % & \textsc{电话:}
    & (+86)15210560972 \\
    % & \textsc{邮箱:}
    & \href{mailto:chen@shijiang.info}{chen@shijiang.info} \\
    % & \textsc{主页:}
    & \href{http://dectinc.cc}{http://dectinc.cc} \\
    % & \textsc{地址:}
    & 北京市清华大学主楼东配楼11-417
\end{tabular}
\\
\\
% \par{\centering
% 		{\Huge 陈仕江 \textsc{}
% 	}\bigskip
% \par}

% %--------------------SECTIONS-----------------------------------
% %Section: Personal Data
% \section{个人信息}

% \begin{tabular}{rlrl}
%     \textsc{电话:} & (+86)15210560972 &
%     \textsc{邮箱:} & \href{mailto:chen@shijiang.info}{chen@shijiang.info} \\
%     \textsc{~~~~~地址:} & 北京市海淀区清华大学主楼东配楼11-417 &
%     \textsc{~~~~~主页:} & \href{http://dectinc.cc}{http://dectinc.cc}
% \end{tabular}

%Section: Education
\section{教育背景}
\begin{tabular}{rl}
 \textsc{2013/8至今} & 清华大学,工学硕士研究生,软件工程专业 \\
 & \small{$\bullet$ 清华大学综合优秀奖学金 ~ $\bullet$ 清华之友-乐逍遥奖学金} \\
 \textsc{2009/8-2013/7} & 清华大学,工学学士,计算机软件专业 \\
 & \small{$\bullet$ 清华大学学业优秀奖学金 ~ $\bullet$ 清华大学软件学院学生科协副主席} \\
 \textsc{2010/8-2013/7} & 清华大学,经济学学士(二学位),经济学专业
\end{tabular}

\section{个人能力}
\begin{tabular}{rl}
 \textsc{~~~~~编程相关}& \textsc{Java}, \textsc{Python}, \textsc{Matlab}, \textsc{Shell}, \textsc{SQL}, \textsc{MongoDB}, \textsc{Redis}\\
 \textsc{~~~~~工具使用}& \textsc{Linux}, \textsc{Latex}, \textsc{SVN}, \textsc{Git}, \textsc{Eclipse}, \textsc{JetBrains系列}\\
\end{tabular}

%Section: Work Experience at the top
\section{项目研究}
% \begin{tabular}{r|p{12cm}}
% \textsc{2015/2至今} & \textbf{基于局部特征与卷积神经网络的图像表示优化} \\
% & \small{$\bullet$ 基于图像中局部特征及CNN学习图像特征表示。通过解特征维度相关及幂式归一化等,减轻特征学习时“丛发词”带来的负面影响} \\
% & \small{$\bullet$ 本人主要使用Matlab、Caffe完成算法实现} \\
% \textsc{2014/4-2015/2} & \textbf{Tetra Fusion Engine} \\
% & \small{$\bullet$ TFE是一个互联网商情实时抓取及分析系统} \\
% & \small{$\bullet$ 独立使用Java开发抓取框架,基于Netty实现任务分发及数据持久化模块} \\
% \textsc{2014/9-2014/11} & \textbf{MIGDemo} \\
% & \small{$\bullet$ 本人主要负责设计并开发的在线多媒体数据分析处理系统,可以实时完成车型检测、事件检测、人体区域检测等任务} \\
% \textsc{2014/7} & \textbf{TransMonitor} \\
% & \small{$\bullet$ 使用Python开发的对神马搜索转码效果进行监控的工具,每日提供相关指标监控报告} \\
% \textsc{2012/7-2013/6} & \textbf{基于主题模型和深度学习的短文本分类} \\
% & \small{$\bullet$ 运用基于多粒度改进的主题模型,引入深度学习方法获取隐含特征,并补充短文本中,改善其特征稀疏、信息量少等问题,提高短文本分类精度}
% \end{tabular}
\begin{entrylist}
  \entry
    {2015/2至今}
    {基于局部特征与卷积神经网络的图像表示优化}
    {研究项目}
    {$\bullet$ 基于图像中局部特征及CNN学习图像特征表示。通过解特征维度相关及幂式归一化等,减轻特征学习时“丛发词”带来的负面影响 \\
    $\bullet$ 本人主要使用Matlab完成算法实现}
  % \entry
  %   {2014/10-2015/3}
  %   {TrecVID MED,大规模多媒体复杂事件检测}
  %   {研究和比赛项目}
  %   {$\bullet$ 使用多种特征及不同融合方式对视频进行描述,结合语义概念的引入及DCNN的使用改善事件检测结果\\
  %   $\bullet$ 本人负责并使用Matlab、Java完成事件检测研究与工具开发}
  % \entry
  %   {2014/9-2014/11}
  %   {保局部稀疏嵌入哈希方法研究(Local Sparse Embedded Hashing)}
  %   {研究项目}
  %   {$\bullet$ LSEH是一种融合稀疏编码与矩阵分解的保数据局部性质的哈希方法\\
  %   $\bullet$ 本人提出及实现LSEH(Matlab版),并在公开数据集上进行充分实验验证}
  \entry
    {2014/4-2015/2}
    {Tetra Fusion Engine}
    {创业项目}
    {$\bullet$ TFE是一个互联网商情实时抓取及分析系统\\
    $\bullet$ 独立使用Java开发抓取框架,基于Netty实现任务分发及数据持久化}
  \entry
    {2014/9-2014/11}
    {MIGDemo}
    {合作项目}
    {$\bullet$ 本人主要负责设计并开发的在线多媒体数据分析处理系统,可以实时完成车型检测、事件检测、人体区域检测等任务}
  \entry
    {2014/7}
    {TransMonitor}
    {实习项目}
    {$\bullet$ 使用Python开发的对神马搜索转码效果进行监控的工具}
  \entry
    {2012/7-2013/6}
    {基于主题模型和深度学习的短文本分类}
    {本科毕业论文}
    {$\bullet$ 运用基于多粒度改进的主题模型,引入深度学习方法获取隐含特征,并补充到短文本中,改善其特征稀疏、信息量少等问题,提高短文本分类精度}
  % \entry
  %   {2013/12-2013/7}
  %   {SocLight}
  %   {合作项目}
  %   {$\bullet$ 本人主要负责实现足球比赛视频关键事件,如角球、进球等的检测应用。}
\end{entrylist}

%Section: Scholarships and additional info
\section{实习经历}
\begin{entrylist}
  \internentry
    {2015/7至今}
    {微软亚洲互联网工程院}
    {实习生}
    {基于大搜索结果补充的图像搜索效果优化}
  \internentry
    {2014/9-2015/7}
    {清华大学}
    {助教}
    {数字媒体2(本科生课程)、软件测试(研究生课程)助教}
  \internentry
    {2014/7-2014/10}
    {神马搜索}
    {实习生}
    {参与转码日常工作,负责开发转码效果监控、黑白名单及配置管理等工具}
  \internentry
    {2013/6-2013/9}
    {阿里巴巴无线搜索事业部}
    {算法工程师(实习)}
    {Hubble抓取系统功能及性能自动化测试}
  \internentry
    {2012/10-2013/3}
    {易思卓科技}
    {Android开发工程师(实习)}
    {SayHi!(千万级用户) Android端更新与维护及若干Android App开发}
\end{entrylist}

\end{document}

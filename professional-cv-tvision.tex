%!TEX TS-program = xelatex

\documentclass[a4paper,10pt]{professional-cv-cn}
\usepackage{fontawesome}

\usepackage{marvosym}
\usepackage{xunicode,xltxtra,url,parskip}
\RequirePackage{color,graphicx}
\usepackage[usenames,dvipsnames]{xcolor}
\usepackage[big]{layaureo}
\usepackage{supertabular}
\usepackage{titlesec}

\usepackage{hyperref}
\definecolor{linkcolour}{rgb}{0,0,0}
\hypersetup{colorlinks,breaklinks,urlcolor=linkcolour, linkcolor=linkcolour}

\titleformat{\section}{\Large\scshape\raggedright}{}{0em}{}[\titlerule]
\titlespacing{\section}{0pt}{2pt}{2pt}
\hyphenation{im-pre-se}

\usepackage[absolute]{textpos}
\usepackage{multirow}
\begin{document}

\pagestyle{empty}

\font\fb=''[cmr10]''

\begin{tabular}{lr}
\multirow{4}{24em}{\Huge ~~~~~~~~~陈仕江}

    & (+86)15210560972 \\

    & \href{mailto:shijiang@aliyun.com}{shijiang@aliyun.com} \\

    & 浙江省杭州市余杭区未来健康科创园 B 幢 1 层
\end{tabular}
\\
\\

\section{教育背景}
\begin{tabular}{rl}
 \textsc{2013/08-2016/07} & 清华大学,工学硕士,软件工程专业 \\
 \textsc{2009/08-2013/07} & 清华大学,工学学士,计算机软件专业 \\
 \textsc{2010/08-2013/07} & 清华大学,经济学学士(二学位),经济学专业
\end{tabular}

\section{个人能力}
% \begin{tabular}{rl}
%  \textsc{\textbf{研究领域}}& \textsc{活体检测}, \textsc{ 图像特征 }\\
%  \textsc{\textbf{编程相关}}& \textsc{Java}, \textsc{Python}, \textsc{Matlab}, \textsc{Shell}\\
%  % \textsc{~~~~~工具使用}& \textsc{Linux}, \textsc{Latex}, \textsc{Git}\\
% \end{tabular}

\begin{entrylist}
  \internentry
    {~~~~~~~~~~\textbf{主要成果}}
    {}
    {}
    {\textsc{有图形晶圆缺陷检测设备}, \textsc{SiC 衬底和外延缺陷检测设备}, \textsc{工业视觉算法研发平台}}
  \internentry
    {~~~~~~~~~~\textbf{研究领域}}
    {}
    {}
    {\textsc{半导体量检测装备}, \textsc{AI 缺陷检测系统}, \textsc{图像特征}}
  \internentry
    {~~~~~~~~~~\textbf{编程相关}}
    {}
    {}
    {\textsc{Java}, \textsc{Python}, \textsc{Matlab}, \textsc{Shell}}
\end{entrylist}

\section{工作经历}
\begin{entrylist}
  \internentry
    {2018/06至今}
    {北京紫睛科技有限公司}
    {全栈工程师}
    {主要负责软件架构设计及开发测试项目管理}
  \internentry
    {2016/06-2018/06}
    {北京搜狐新媒体信息技术有限公司}
    {研发工程师}
    {负责低质过滤系统和个性化推荐引擎研发}
\end{entrylist}

\section{项目研究}

\begin{entrylist}
  \entry
    {2022/08至今}
    {工业视觉算法研发平台开发及测试}
    {}
    {$\bullet$ 负责并组织该研发平台的系统设计、开发与测试工作,并完成某工业视觉头部上市企业交付。 \\
    $\bullet$ 解决图像识别、模型压缩优化、模型在线更新等核心技术点,支持印刷、电子屏幕等多场景AI模型研发流程,实现数亿元产值转化。}
  \entry
    {2018/08-2020/07}
    {某大型政务系统核心子系统}
    {}
    {$\bullet$ 主导并完成了该核心子系统的设计与核心模块开发,采用 “大平台微服务” 架构,提高了系统的灵活性和业务响应速度。\\
    $\bullet$ 沉淀了系统迁移、编码开发和开发流程规范,为后续信息化建设提供了标准化技术框架。}
  \entry
    {2018/06-2018/08}
    {无感人脸识别系统及活体检测算法}
    {}
    {$\bullet$ 独立负责多视频源输入、支持多种不同算法检测及报警提醒的无感人脸识别系统,在 i5U、8G 内存的标准 NUC 上,实现准实时三路无感人脸识别及控制。 \\
    $\bullet$ 独立负责研究并开发活体检测算法,活体检测准确率提升10.8\%,活体检测效率提升35\%。}
  \entry
    {2016/08-2018/05}
    {搜狐新闻推荐引擎}
    {}
    {$\bullet$ 本人独立负责设计并主要参与研发的搜狐门户网站新闻推荐引擎,提供融合专家推荐与个性化推荐结果的实时新闻推荐。 \\
    $\bullet$ 推荐引擎支持多算法、实验配置,ABTest 等功能;上线后各频道转化率平均提高6\%。}
\end{entrylist}

\end{document}

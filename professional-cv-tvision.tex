%!TEX TS-program = xelatex
\documentclass[a4paper,10pt]{professional-cv-cn}

% ===== 宏包加载 =====
\usepackage{fontawesome,marvosym,xunicode,xltxtra,url,parskip}
\usepackage{color,graphicx,xcolor}
\usepackage[big]{layaureo}
\usepackage{supertabular,titlesec,hyperref,multirow,textpos}

% ===== 配置管理 =====
\definecolor{linkcolour}{rgb}{0,0,0}
\hypersetup{
    colorlinks,
    breaklinks,
    urlcolor=linkcolour, 
    linkcolor=linkcolour
}

% ===== 标题格式 =====
\titleformat{\section}{\Large\scshape\raggedright}{}{0em}{}[\titlerule]
\titlespacing{\section}{0pt}{8pt}{4pt}
\hyphenation{im-pre-se}

\begin{document}
\pagestyle{empty}

% ===== 个人信息 =====
\begin{tabular}{@{} l r @{}}
    \multirow{3}{19.5em}{\Huge ~~~~~~~~~陈仕江}
        & (+86)15210560972 \\
        & \href{mailto:shijiang@aliyun.com}{shijiang@aliyun.com} \\
        & 浙江省杭州市余杭区未来健康科创园 B 幢 1 层
\end{tabular}

% ===== 教育背景 =====
\section{教育背景}
\begin{tabular}{r l}
    \textsc{2013/08-2016/07}
        & 清华大学,工学硕士,软件工程专业
    \\
    \textsc{2009/08-2013/07}
        & 清华大学,工学学士,计算机软件专业
    \\
    \textsc{2010/08-2013/07}
        & 清华大学,经济学学士(二学位),经济学专业
\end{tabular}

% ===== 核心能力 =====
\section{核心能力}
\begin{entrylist}
    \internentry
        {~~~~~~~~~~\textbf{技术方向}}
        {}
        {}
        {
            \textsc{半导体量检测装备研发},
            \textsc{工业视觉算法架构},
            \textsc{AI缺陷检测系统}
        }
    \internentry
        {~~~~~~~~~~\textbf{管理专长}}
        {}
        {}
        {
            \textsc{技术团队建设},
            \textsc{产品战略规划},
            \textsc{装备生产及交付},
            \textsc{科技成果转化}
        }
\end{entrylist}

% ===== 工作经历 =====
\section{工作经历}
\begin{entrylist}
    \entry
        {2021/06至今}
        {清软微视(杭州)科技有限公司}
        {联合创始人/副总经理}
        {
            $\bullet$ \textbf{产品研发}:主导开发 Alpha 系列有图形晶圆检测装备与 Omega 系列 SiC 衬底/外延检测装备,技术参数在同类型应用达国际领先水平,稳定交付数十台。
            \\
            $\bullet$ \textbf{技术管理}:综合管理光学/机械/电气/软件/算法全栈研发团队,以及数据/生产/测试/文档/现场全流程技术支持团队,覆盖产品售前/设计/研发/生产/交付全周期。
            \\
            $\bullet$ \textbf{平台建设}:自主研发适用于半导体缺陷检测等工业视觉场景的算法平台,实现70\%人工复检替代,支撑半导体全制程缺陷分析。
            \\
            $\bullet$ \textbf{产业化落地}:实现全自主知识产权 SiC 缺陷检测装备量产交付,打破国外垄断。
        }
    \entry
        {2020/11-2021/05}
        {清华大学}
        {科研工程师}
        {
            $\bullet$ 负责并组织工业视觉算法研发平台的系统设计与开发,完成某工业视觉头部上市企业交付,实现数亿元产值转化。
        }
    \entry
        {2018/09-2020/10}
        {北京紫睛科技有限公司/杭州广目科技有限公司}
        {软件架构师}
        {
            $\bullet$ 主导并完成某大型政务系统核心子系统的设计与核心模块开发。
            \\
            $\bullet$ 主导无感测温、动态视频流人脸识别等系统级产品研发,在国家电网、高铁站等场景实现部署应用。
        }
    \justentry
        {2016/08-2018/08}
        {搜狐新媒体信息技术有限公司}
        {研发工程师}
\end{entrylist}

% ===== 关键成果 =====
\section{关键成果}
\begin{entrylist}
    \entry
        {~~~~~~~~~2023-2025}
        {Alpha~系列有图形晶圆检测装备}
        {}
        {
            $\bullet$ 集成大视场高分辨多通道成像与~AI~算法,支持最大至 12 吋多种半导体晶圆检测。
            \\
            $\bullet$ 检测效率提升30\%,漏判率<0.01\%,达行业顶尖水平。
        }
    \entry
        {~~~~~~~~~2022-2024}
        {Omega~系列~SiC~衬底/外延检测装备}
        {}
        {
            $\bullet$ 突破\textbf{多模态成像技术}(明场/暗场/PL),实现亚微米级缺陷精准检测与分类。
            \\
            $\bullet$ 相比国外装备,提升检测精度 3 倍,PL 信号强度 2 倍,于多家头部 SiC 企业验收。
        }
    \entry
        {~~~~~~~~~2021-2023}
        {AIRS 工业视觉算法平台}
        {}
        {
            $\bullet$ 集成深度神经网络及主动学习框架,开发 AI 复判系统,替代~70\%人工复检。
            \\
            $\bullet$ 适配~KLA/Camtek~等主流机台,服务半导体全制程。
        }
\end{entrylist}
\end{document}
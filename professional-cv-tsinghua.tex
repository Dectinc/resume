%!TEX TS-program = xelatex

\documentclass[a4paper,10pt]{professional-cv-cn}
\usepackage{fontawesome}

\usepackage{marvosym}
\usepackage{xunicode,xltxtra,url,parskip}
\RequirePackage{color,graphicx}
\usepackage[usenames,dvipsnames]{xcolor}
\usepackage[big]{layaureo}
\usepackage{supertabular}
\usepackage{titlesec}

\usepackage{hyperref}
\definecolor{linkcolour}{rgb}{0,0,0}
\hypersetup{colorlinks,breaklinks,urlcolor=linkcolour, linkcolor=linkcolour}

\titleformat{\section}{\Large\scshape\raggedright}{}{0em}{}[\titlerule]
\titlespacing{\section}{0pt}{2pt}{2pt}
\hyphenation{im-pre-se}

\usepackage[absolute]{textpos}
\usepackage{multirow}
\begin{document}

\pagestyle{empty}

\font\fb=''[cmr10]''

\begin{tabular}{lr}
\multirow{4}{24em}{\Huge ~~~~~~~~~陈仕江}

    & (+86)15210560972 \\

    & \href{mailto:tsinghua@dectinc.cc}{tsinghua@dectinc.cc} \\

    & 北京市海淀区中关村智造大街 A 座
\end{tabular}
\\
\\

\section{教育背景}
\begin{tabular}{rl}
 \textsc{2013/08-2016/07} & 清华大学,工学硕士,软件工程专业 \\
 \textsc{2009/08-2013/07} & 清华大学,工学学士,计算机软件专业 \\
 \textsc{2010/08-2013/07} & 清华大学,经济学学士(二学位),经济学专业
\end{tabular}

\section{个人能力}
% \begin{tabular}{rl}
%  \textsc{\textbf{研究领域}}& \textsc{活体检测}, \textsc{ 图像特征 }\\
%  \textsc{\textbf{编程相关}}& \textsc{Java}, \textsc{Python}, \textsc{Matlab}, \textsc{Shell}\\
%  % \textsc{~~~~~工具使用}& \textsc{Linux}, \textsc{Latex}, \textsc{Git}\\
% \end{tabular}

\begin{entrylist}
  \internentry
    {~~~~~~~~~~\textbf{专业技能}}
    {}
    {}
    {\textsc{系统设计与研发}, \textsc{计算机视觉算法产品化}, \textsc{活体检测}, \textsc{图像特征}}
  \internentry
    {~~~~~~~~~~\textbf{编程相关}}
    {}
    {}
    {\textsc{Java}, \textsc{Python}, \textsc{Vue.js}, \textsc{Docker}, \textsc{Spring MVC}, \textsc{Spring Cloud}, \textsc{Tornado}, \textsc{FastAPI}}
\end{entrylist}

\section{工作经历}
\begin{entrylist}
  \internentry
    {2018/07至今}
    {北京紫睛科技有限公司}
    {全栈工程师}
    {负责视频智能分析系统研发及无感人脸识别系统研发}
  \internentry
    {2016/06-2018/07}
    {北京搜狐新媒体信息技术有限公司}
    {研发工程师}
    {负责低质过滤系统和个性化推荐引擎研发}
  \internentry
    {2014/09-2015/07}
    {清华大学}
    {助教}
    {数字媒体2(本科生课程)、软件测试(研究生课程)助教}
\end{entrylist}

\section{项目研究}

\begin{entrylist}
  \entry
    {2019/11至今}
    {Argus 视频智能分析系统}
    {}
    {$\bullet$ 本系统可支持同时对万路级在线视频进行管理和分析,每路在线视频可同时进行多种视觉分析\\
    $\bullet$ 主要负责系统设计和统计分析服务研发,基于单机版本研发的测温系统已在某特大型国有企业现场投入使用}
  \entry
    {2018/11至2019/10}
    {新一代通关系统电子审单模块}
    {}
    {$\bullet$ 本系统需要对目前海关线上 2000 年开发的通关系统进行在线改造和技术优化\\
    $\bullet$ 需求设计和系统研发主要负责人,组织最多 30 人开发团队同时工作,系统于 2019 年 8 月上线投入使用}
  \entry
    {2018/06-2018/10}
    {无感人脸识别系统}
    {}
    {$\bullet$ 独立负责研发读取多路摄像机视频源,检测并与底库匹配视频源中的人脸,并根据配置进行多种不同报警操作的无感人脸识别系统\\
    $\bullet$ 在 i5U、8G 内存的标准 NUC 上,实现准实时三路无感人脸识别及控制}
  \entry
    {2018/03-2018/05}
    {搜狐新闻推荐引擎}
    {}
    {$\bullet$ 本人独立负责设计并主要参与研发的搜狐门户网站新闻推荐引擎,提供融合专家推荐与个性化推荐结果的实时新闻推荐 \\
    $\bullet$ 推荐引擎支持多算法、实验配置,ABTest 等功能;上线后各频道转化率平均提高6\%}
  \entry
    {2016/08-2017/02}
    {搜狐新闻低质过滤系统}
    {}
    {$\bullet$ 独立负责低质新闻识别算法研究及低质过滤系统开发\\
    $\bullet$ 低质过滤系统各频道平均低质判断准确率92\%,每日减少人工50\%工作量}
\end{entrylist}

\end{document}
